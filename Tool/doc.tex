\documentclass[a4paper,12pt]{article}
\usepackage[utf8]{inputenc}
\usepackage[T1]{fontenc}
\usepackage{lmodern}
\usepackage{hyperref}
\usepackage{listings}
\usepackage{xcolor}
\usepackage{amsmath}

\definecolor{backcolour}{rgb}{0.95,0.95,0.92}
\definecolor{codegray}{rgb}{0.5,0.5,0.5}
\definecolor{codepurple}{rgb}{0.58,0,0.82}
\definecolor{keywordcolor}{rgb}{0,0,1}

\lstdefinestyle{mystyle}{
backgroundcolor=\color{backcolour},
commentstyle=\color{codegray},
keywordstyle=\color{keywordcolor},
numberstyle=\tiny\color{codegray},
stringstyle=\color{codepurple},
basicstyle=\ttfamily\footnotesize,
breakatwhitespace=false,
breaklines=true,
captionpos=b,
keepspaces=true,
numbers=left,
numbersep=5pt,
showspaces=false,
showstringspaces=false,
showtabs=false,
tabsize=2
}

\lstset{style=mystyle}

\title{Documentazione dello Script per la Classificazione dei Prompt e l'Associazione di CWE}
\author{Il Tuo Nome}
\date{\today}

\begin{document}

\maketitle

\begin{abstract}
Questo documento descrive uno script Python per la classificazione di prompt e l'associazione di vulnerabilit`a CWE. Lo script utilizza modelli di embedding della famiglia Sentence-Transformers e un modello di regressione Random Forest per predire una probabilit`a complessiva di vulnerabilit`a per ciascun prompt. Inoltre, per ogni nuovo prompt vengono identificate fino a tre candidate CWE, con le rispettive probabilit`a normalizzate, e i risultati sono salvati in un file CSV.
\end{abstract}

\section{Calcolo della Probabilit`a delle CWE}

Per ciascun nuovo prompt, viene calcolata la similarità coseno tra il suo embedding e gli embeddings presenti nel dataset:

\begin{equation}
\text{similarit`a}(A, B) = \frac{A \cdot B}{||A|| ||B||}
\end{equation}

dove $A$ e $B$ sono i vettori di embedding dei prompt.

Le CWE candidate vengono selezionate in base alla similarit`a pi`u alta, e la probabilit`a normalizzata per ciascuna CWE `e calcolata come:

\begin{equation}
P(CWE_i) = \frac{\text{similarit`a}(P, D_i)}{\sum_{j=1}^{k} \text{similarit`a}(P, D_j)}
\end{equation}

dove:
\begin{itemize}
\item $P$ `e il prompt in input,
\item $D_i$ `e il dataset dei prompt con la CWE associata,
\item $k$ `e il numero massimo di CWE candidate (3 per default).
\end{itemize}

Se un prompt ha meno di tre candidate CWE, i campi mancanti nel CSV vengono riempiti con "None".

\end{document}